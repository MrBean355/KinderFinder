\documentclass[11pt,titlepage]{article} % use larger type; default would be 10pt

\usepackage[utf8]{inputenc} % set input encoding (not needed with XeLaTeX)

%%% PAGE DIMENSIONS
\usepackage{geometry} % to change the page dimensions
\geometry{a4paper} % or letterpaper (US) or a5paper or....
\geometry{margin=2cm} % for example, change the margins to 2 inches all round
\geometry{portrait} % set up the page for landscape
%   read geometry.pdf for detailed page layout information

\usepackage{graphicx} % support the \includegraphics command and options

\usepackage[parfill]{parskip} % Activate to begin paragraphs with an empty line rather than an indent

%%% PACKAGES
%\usepackage{booktabs} % for much better looking tables
%\usepackage{array} % for better arrays (eg matrices) in maths
%\usepackage{paralist} % very flexible & customisable lists (eg. enumerate/itemize, etc.)
\usepackage{verbatim} % adds environment for commenting out blocks of text & for better verbatim
%\usepackage{subfig} % make it possible to include more than one captioned figure/table in a single float
% These packages are all incorporated in the memoir class to one degree or another...
%\usepackage{glossaries} %for the glossary

%%% HEADERS & FOOTERS
%\usepackage{fancyhdr} % This should be set AFTER setting up the page geometry
%\pagestyle{fancy} % options: empty , plain , fancy
%\renewcommand{\headrulewidth}{0pt} % customise the layout...
%\lhead{}\chead{}\rhead{}
%\lfoot{}\cfoot{\thepage}\rfoot{}

 
%%% END Article customizations

%%% The "real" document content comes below...

%%% Maketitle metadata
\newcommand{\horrule}[1]{\rule{\linewidth}{#1}} 	% Horizontal rule

\title{
		\normalfont \normalsize \textsc{Client Name: DVT} \\
		\normalfont \normalsize \textsc{Project Name: KinderFinder} \\ [25pt]
		\horrule{0.5pt} \\[0.4cm]
		\huge Software Architecture Documentation \\
		\horrule{2pt} \\[0.5cm]
}
\author{\begin{tabular}{rl}
	\texttt{Team Name:} & \texttt{MAU Technologies} \\[0.5cm]
	Uteshlen Nadesan & 28163304 \\
	Michael Johnston & 12053300 \\
	Po-Han Chiu & 11063612
\end{tabular}
	\\ \\ \texttt{https://github.com/MrBean355/KinderFinder.git}
	\\ \\ \texttt{Version: 0.0}}
\date{23 May 2014}  
% Activate to display a given date or no date (if empty),
         % otherwise the current date is printed -------

%%% Begin document
\begin{document}
\maketitle
\tableofcontents
\newpage

%4.1
\section{Architecture requirements}
% In this section extract the architectural requirements from the software requirements including
% \\-scope of architectural responsibilities (e.g. persistence, reporting, process execution, . . . ),
% \\-quantified quality requirements,
% \\-integration and access channel requirements, and
% \\-any architectural constraints.

%4.1.1
\subsection{Architectural scope}
The project will consist of five main components of which one is still a proof of concept (POC).
\begin{itemize}
	\item{Web Administration}
	\item{Mobile Application}
	\item{Web API}
	\item{Basic Reporting}
	\item{Embedded Hardware and prototyping (POC)}
	\end{itemize}

%%%4.1.2
%\subsection{Quality requirements}
% This section should state the quality requirements in order of priority/importance. Examples of
% quality requirements are scalability, reliability, performance, security, audit-ability, integrability, . . . .
% Each quality requirement needs to be quantified. For example, scalability could be specified in
% terms of number of transactions per unit time or number of concurrent users.

%%%4.1.3
%\subsection{Integration and access channel requirements}
% In this section you need to specify the different system which your system must integrate with and
% the integration channels and protocols which need to be used.
% You should also specify the different access channels through which the system functionality
% should be made available to humans and/or other systems.

%4.1.4
\subsection{Architectural constraints}
The only constraint for this project is the tools used to build the database for the system, those are the tools the client prefers and specified in the requirements specification.

%%%4.2
%\section{Architectural patterns or styles}
% In this section discuss any architectural patterns or styles you have chosen to use together with the
% rationale for using them.
% For example, you might have decided to use layering in order to have lower level layers reusable
% across different higher level layers (e.g. have a services layer usable by a web front-end and a
% mobile-device client).
% Discuss the elements of the architectural pattern (e.g. the different layers) and how these elements are connected (e.g. the integration channels between the layers).
%%%MVC

%%%4.3
%\section{Architectural tactics or strategies}
% In this section discuss any architectural tactics or strategies you are using to concretely address any
% of the quality requirements.
% For example, you could be using thread pooling and/or caching to achieve a higher level of
% scalability or performance.
% Discuss how the strategy is realized within your software architecture.

%%%4.4
%\section{Use of reference architectures and frameworks}
% In this section discuss any reference architectures and/or frameworks you might be incorporating
% within your software architecture. For example, you could be using an object-relational mapper or
% a particular adapter or even an application server or an enterprise services bus.
% Discuss the reasons for choosing such frameworks for your software architecture.

%4.5
\section{Access and integration channels}
End product users will access the system through a Mobile application and the administrators of the system will access the system through a Web API with more privileges and higher level of access of the system compared to the average user. Due to the nature of the project and technologies that will be used, integration is fairly due to the fact that the programming languages for most components are all compatible.

%4.6
\section{Technologies}
The technologies that will be used  for this project are the following:
\begin{itemize}
	\item{Microsoft Visio Studio}
	\item{ASP .Net MVC 5}
	\item{HTML 5}
	\item{JavaScript}
	\item{JQuery}
	\item{ASP .Net WebApi}
	\item{C\#}
	\item{Microsoft SQL Server 2012}
	\item{Xamarin}
	\end{itemize}
The above mentioned technologies are the technologies that are to be used for this project, but may still be subject to change at a later stage of the software development due to the unforeseen situations. 
	

\newpage
\appendix
\section{Glossary} \label{App:AppendixA}
% the \\ insures the section title is centered below the phrase: AppendixA
\begin{tabular}{rl}
	\textbf{Acronym} & \textbf{Definition} \\[0.5cm]
	ASP .NET MVC & A Project type in Visual Studio implementing MVC pattern \\
	MVC & Model View Controller \\
	POC & Proof of Concept
\end{tabular}

\end{document}
