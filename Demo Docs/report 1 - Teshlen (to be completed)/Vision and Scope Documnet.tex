%%%
%%  Percentage signs are comments.
%%

% This is a simple template for a LaTeX document using the "article" class.
% See "book", "report", "letter" for other types of document.

\documentclass[11pt,titlepage]{article} % use larger type; default would be 10pt

\usepackage[utf8]{inputenc} % set input encoding (not needed with XeLaTeX)

%%% PAGE DIMENSIONS
\usepackage{geometry} % to change the page dimensions
\geometry{a4paper} % or letterpaper (US) or a5paper or....
\geometry{margin=2cm} % for example, change the margins to 2 inches all round
\geometry{portrait} % set up the page for landscape
%   read geometry.pdf for detailed page layout information

\usepackage{graphicx} % support the \includegraphics command and options

\usepackage[parfill]{parskip} % Activate to begin paragraphs with an empty line rather than an indent

%%% PACKAGES
%\usepackage{booktabs} % for much better looking tables
%\usepackage{array} % for better arrays (eg matrices) in maths
%\usepackage{paralist} % very flexible & customisable lists (eg. enumerate/itemize, etc.)
\usepackage{verbatim} % adds environment for commenting out blocks of text & for better verbatim
%\usepackage{subfig} % make it possible to include more than one captioned figure/table in a single float
% These packages are all incorporated in the memoir class to one degree or another...
%\usepackage{glossaries} %for the glossary

%%% HEADERS & FOOTERS
%\usepackage{fancyhdr} % This should be set AFTER setting up the page geometry
%\pagestyle{fancy} % options: empty , plain , fancy
%\renewcommand{\headrulewidth}{0pt} % customise the layout...
%\lhead{}\chead{}\rhead{}
%\lfoot{}\cfoot{\thepage}\rfoot{}

%%% SECTION TITLE APPEARANCE
%\usepackage{sectsty}
%\allsectionsfont{\sffamily\mdseries\upshape} % (See the fntguide.pdf for font help)
% (This matches ConTeXt defaults)

%%% ToC (table of contents) APPEARANCE
%\usepackage[nottoc,notlof,notlot]{tocbibind} % Put the bibliography in the ToC
%\usepackage[titles,subfigure]{tocloft} % Alter the style of the Table of Contents
%\renewcommand{\cftsecfont}{\rmfamily\mdseries\upshape}
%\renewcommand{\cftsecpagefont}{\rmfamily\mdseries\upshape} % No bold!

%%% END Article customizations

%%% The "real" document content comes below...

%%% Maketitle metadata
\newcommand{\horrule}[1]{\rule{\linewidth}{#1}} 	% Horizontal rule

\title{
		\normalfont \normalsize \textsc{Client Name: DVT} \\
		\normalfont \normalsize \textsc{Project Name: KinderFinder} \\ [25pt]
		\horrule{0.5pt} \\[0.4cm]
		\huge Vision and Scope Document \\
		\horrule{2pt} \\[0.5cm]
}
\author{\begin{tabular}{rl}
	\texttt{Team Name:} & \texttt{MAU Technologies} \\[0.5cm]
	Uteshlen Nadesan & 28163304 \\
	Michael Johnston & 12053300 \\
	Po-Han Chiu & 11063612
\end{tabular}
	\\ \\ \texttt{https://github.com/MrBean355/KinderFinder.git}
	\\ \\ \texttt{Version: 0.0}}
\date{23 May 2014} 

% Activate to display a given date or no date (if empty),
% otherwise the current date is printed -------

%%% Begin document
\begin{document}
\maketitle
\tableofcontents
\newpage

The document is developed at the beginning of the project, but can be modified and extended as the vision and scope for the project evolve.
A description of the vision of the project. This typically includes the main purpose of the
project and what the client aims to achieve with the project ( i.e. the business or user needs the project aims to address).
The document often discusses what the alternatives are, e.g. how the functionality is done
without the system and/or what the deficiencies of alternative products are.

%2.1
\section{Scope and Limitations/Exclusions}
The document should specify the major features, high level scope and limitations of the proposed software product.
\\ Use a high-level use case diagram with
\\ -abstract use cases for services/responsibility domains of the system,
\\ -concrete use cases (the leaf use cases in the specialization hierarchy) as the required concrete use cases/user services,
\\ -optionally some include and extend relationships to show the core functional requirements, and
\\ -actors showing the external systems which are not part of the scope of the system, but which the system integrates with.
\\ List the exclusions/limitations, discussing any functionality which could erroneously be assumed within scope but which has been explicitly excluded from the scope of the system.

\newpage
\appendix
\section{Glossary} \label{App:AppendixA}
% the \\ insures the section title is centered below the phrase: AppendixA

Text of Glossary is Here
\end{document}
